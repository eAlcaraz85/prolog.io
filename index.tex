% Created 2023-02-27 lun 09:46
% Intended LaTeX compiler: pdflatex
\documentclass[11pt]{article}
\usepackage[utf8]{inputenc}
\usepackage[T1]{fontenc}
\usepackage{graphicx}
\usepackage{grffile}
\usepackage{longtable}
\usepackage{wrapfig}
\usepackage{rotating}
\usepackage[normalem]{ulem}
\usepackage{amsmath}
\usepackage{textcomp}
\usepackage{amssymb}
\usepackage{capt-of}
\usepackage{hyperref}
\usepackage{minted}
\usepackage[spanish]{inputenc}
\author{Eduardo Alcaraz}
\date{\today}
\title{Apuntes Programación Lógica Funcional}
\hypersetup{
 pdfauthor={Eduardo Alcaraz},
 pdftitle={Apuntes Programación Lógica Funcional},
 pdfkeywords={},
 pdfsubject={},
 pdfcreator={Emacs 27.1 (Org mode 9.3)}, 
 pdflang={English}}
\begin{document}

\maketitle
\tableofcontents


\section*{Estilos de Programación}
\label{sec:orgfa7c346}

(también llamados estándares de código o convención de código) es un
término que describe convenciones para escribir código fuente en
ciertos lenguajes de programación. El estilo de programación es
frecuentemente dependiente del lenguaje de programación que se haya
elegido para escribir. Por ejemplo el estilo del lenguaje de
Programación C variará con respecto al del lenguaje BASIC.


\subsection*{Características del Estilo}
\label{sec:org3aee4de}

Una pieza clave para un buen estilo es la elección apropiada de
nombres de variable. Variables pobremente nombradas dificultan la
lectura del código fuente y su comprensión.

Como ejemplo, considérese el siguiente extracto de pseudocódigo:

\begin{minted}[]{c}
get a b c 
if a < 24 and b < 60 and c < 60
  return true
else
  return false
\end{minted}

Debido a la elección de nombres de variable, es difícil darse cuenta
de la función del código. Compárese ahora con la siguiente versión:


\begin{minted}[]{c}
 get horas minutos segundos 
 if horas < 24 and minutos < 60 and segundos < 60
   return true
 else
   return false
\end{minted}

La intención el código es ahora más sencilla de discernir, "dado una
hora en 24 horas, se devolverá true si es válida y false si no".


\subsection*{Nombres de Variable Apropiadas.}
\label{sec:orgb39e578}
Una piedra clave para un buen estilo es la elección apropiada de
nombres de variable. Variables pobremente nombradas dificultan la
lectura del código fuente y su comprensión.  y Como ejemplo,
considérese el siguiente extracto de pseudocódigo:

\begin{minted}[]{c}
get a b c
if a < 24 and b < 60 and c < 60
  return true
else
  return false
\end{minted}

Debido a la elección de nombres de variable, es difícil darse cuenta
de la función del código. Compárese ahora con la siguiente versión:

\begin{minted}[]{c}
 get horas minutos segundos 
 if horas < 24 and minutos < 60 and segundos < 60
   return true
 else
   return false
\end{minted}

La intención el código es ahora más sencilla de discernir, "dado una
hora en 24 horas, se devolverá true si es válida y false si no".

\subsection*{Estilo de indentación}
\label{sec:org766c5dc}

Estilo de indentación, en lenguajes de programación que usan llaves
para indentar o delimitar bloques lógicos de código, como por
ejemplo C, es también un punto clave el buen estilo. Usando un
estilo lógico y consistente hace el código de uno más
legible. Compárese:


\begin{minted}[]{c}
 if(horas < 24 && minutos < 60 && segundos < 60){
    return true;
 }else{
    return false;
 }
\end{minted}
o bien:
\begin{minted}[]{c}
 if(horas < 24 && minutos < 60 && segundos < 60)
 {
    return true;
 }
 else
 {
    return false;
 }
\end{minted}

con algo como:

\begin{minted}[]{c}

 if(horas<24&&minutos<60&&segundos<60){return true;}
 else{return false;}
\end{minted}

Los primeros dos ejemplos son mucho más fáciles de leer porque están
bien indentados, y los bloques lógicos de código se agrupan y se
representan juntos de forma más clara.

Valores booleanos en estructuras de decisión Algunos programadores
piensan que las estructuras de decisión como las anteriores, donde el
resultado de la decisión es meramente una computación de un valor
booleano, son demasiado prolijos e incluso propensos al
error. Prefieren hacer la decisión en la computación por sí mismo,
como esto:

\begin{minted}[]{c}
 return horas < 12 && minutos < 60 && segundos < 60;
\end{minted}

La diferencia es, con frecuencia, puramente estilística y sintáctica,
ya que los compiladores modernos producirán código objeto idéntico en
las dos formas.

\subsection*{Bucles y estructuras de control}
\label{sec:org41e073b}

El uso de estructuras de control lógicas para bucles también es parte
 de un buen estilo de programación. Ayuda a alguien que esté leyendo
 el código a entender la secuencia de ejecución (en programación
 imperativa). Por ejemplo, el siguiente pseudocódigo:
\begin{minted}[]{c}

   cuenta = 0
   while cuenta < 5
     print cuenta * 2
     cuenta = cuenta + 1
   endwhile
\end{minted}

El extracto anterior cumple con las dos recomendaciones de estilo
anteriores, pero el siguiente uso de la construcción for hace el
código mucho más fácil de leer:

\begin{minted}[]{c}

 for cuenta = 0, cuenta < 5, cuenta=cuenta+1
   print cuenta * 2
\end{minted}

En muchos lenguajes, el patrón frecuentemente usado "por cada elemento
en un rango" puede ser acortado a:

\begin{minted}[]{c}

 for cuenta = 0 to 5
   print cuenta * 2
\end{minted}

\subsection*{Espaciado}
\label{sec:org47fa7f3}

Los lenguajes de formato libre ignoran frecuentemente los espacios
en blanco. El buen uso del espaciado en la disposición del código
de uno es, por tanto, considerado un buen estilo de programación.

Compárese el siguiente extracto de código C:

\begin{minted}[]{c}
 int cuenta; for(cuenta=0;cuenta<10;cuenta++)
{printf("%d",cuenta*cuenta+cuenta);}
\end{minted}
con:
\begin{minted}[]{c}
 int cuenta;
 for (cuenta = 0; cuenta < 10; cuenta++)
 {
    printf("%d", cuenta * cuenta + cuenta);
 }
\end{minted}

En los lenguajes de programación de la familia C se recomienda también
evitar el uso de caracteres tabulador en medio de una línea, ya que
diferentes editores de textos muestran su anchura de forma diferente.

El lenguaje de programación Python usa indentación para indicar
estructuras de control, por tanto se requiere obligatoriamente una
buena indentación. Haciendo esto, la necesidad de marcar con llaves (\{
y \}) es eliminada, y la legibilidad es mejorada sin interferir con los
estilos de codificación comunes.


Con todo, esto lleva frecuentemente a problemas donde el código es
copiado y pegado dentro de un programa Python, requiriendo un
tedioso reformateado. Adicionalmente, el código Python se vuelve
inusable cuando es publicado en un foro o página web que elimine el
espacio en blanco.

\subsection*{Evaluación de expresiones}
\label{sec:orga1a7dbc}

En general, salvo que se relacionen con las mencionadas sentencias modificadoras del 
flujo, las palabras-clave señalan al compilador aspectos complementarios que no
alteran el orden de ejecución dentro de la propia sentencia. Este orden viene 
determinado por cuatro condicionantes:

\begin{enumerate}
\item Presencia de paréntesis que obligan a un orden de evaluación
específico.
\item Naturaleza de los operadores involucrados en la expresión
(asociatividad).
\item Orden en que están colocados (precedencia).
\item Providencias (impredecibles) del compilador relativas a la
optimización del código.
\end{enumerate}



\section*{Paradigma de Programación}
\label{sec:org3aa0ca3}

Un paradigma de programación es una propuesta tecnológica adoptada por
una comunidad de programadores y desarrolladores cuyo núcleo central
es incuestionable en cuanto que únicamente trata de resolver uno o
varios problemas claramente delimitados; la resolución de estos
problemas debe suponer consecuentemente un avance significativo en al
menos un parámetro que afecte a la ingeniería de software.


\subsection*{Tipos más comunes de paradigmas de programación}
\label{sec:org14b2bce}

\subsubsection*{Programación imperativa o por procedimientos:}
\label{sec:org6a7a66b}
 Es el más usado en general, se basa en dar instrucciones al
ordenador de como hacer las cosas en forma de algoritmos. La
programación imperativa es la más usada y la más antigua, el
ejemplo principal es el lenguaje de máquina. Ejemplos de lenguajes
puros de este paradigma serían el C, BASIC o Pascal.

\subsubsection*{Programación orientada a objetos:}
\label{sec:org3138b89}
Está basada en el imperativo, pero encapsula elementos denominados
objetos que incluyen tanto variables como funciones. Está
representado por C++, C\#, Java o Python entre otros, pero el más
representativo sería el Smalltalk que está completamente orientado a
objetos.


\subsubsection*{Programación dinámica:}
\label{sec:org61208de}
 está definida como el proceso de romper
problemas en partes pequeñas para analizarlos y resolverlos de forma
lo más cercana al óptimo, busca resolver problemas en O(n) sin usar
por tanto métodos recursivos. Este paradigma está más basado en el
modo de realizar los algoritmos, por lo que se puede usar con
cualquier lenguaje imperativo.

\subsubsection*{Programación dirigida por eventos:}
\label{sec:org393daa4}
la programación dirigida por
eventos es un paradigma de programación en el que tanto la estructura
como la ejecución de los programas van determinados por los sucesos
que ocurran en el sistema, definidos por el usuario o que ellos
mismos provoquen.


\subsubsection*{Programación declarativa:}
\label{sec:orga58785c}
 está basado en describir el problema
declarando propiedades y reglas que deben cumplirse, en lugar de
instrucciones. Hay lenguajes para la programación funcional, la
programación lógica, o la combinación lógico-funcional. Unos de los
primeros lenguajes funcionales fueron Lisp y Prolog.

\subsubsection*{Programación funcional:}
\label{sec:org91d9e74}

La programación funcional es un paradigma de programación declarativa
basado en el uso de verdaderas funciones matemáticas. En este estilo
de programación las funciones son ciudadanas de primera clase, porque
sus expresiones pueden ser asignadas a variables como se haría con
cualquier otro valor; además de que pueden crearse funciones de orden
superior.​

La programación funcional tiene sus raíces en el cálculo lambda, un
sistema formal desarrollado en los años 1930 para investigar la
naturaleza de las funciones, la naturaleza de la computabilidad y su
relación con la recursión. Los lenguajes funcionales priorizan el uso
de recursividad y aplicación de funciones de orden superior para
resolver problemas que en otros lenguajes se resolverían mediante
estructuras de control (por ejemplo, ciclos). Algunos lenguajes
funcionales también buscan eliminar la mutabilidad o efectos
secundarios; en contraste con la programación imperativa, que se basa
en los cambios de estado mediante la mutación de variables. Esto
significa que, en programación funcional pura, dos o más expresiones
sintácticas idénticas (por ejemplo, dos llamadas a rutinas o dos
evaluaciones) siempre devolverán el mismo resultado. Es decir, se
tiene transparencia referencial. Lo anterior también puede ser
aprovechado para diseñar estrategias de evaluación que eviten repetir
el cómputo de expresiones antes vistas, ahorrando tiempo de ejecución.

Los lenguajes de programación funcional, especialmente los puramente
funcionales, han sido enfatizados en el ambiente académico y no tanto
en el desarrollo comercial o industrial. Sin embargo, lenguajes de
programación funcional como Lisp (Scheme, Common Lisp, etc.), Erlang,
Rust, Objective Caml, Scala, F\# y Haskell, han sido utilizados en
aplicaciones comerciales e industriales por muchas
organizaciones. También es utilizada en la industria a través de
lenguajes de dominio específico como R (estadística), Mathematica
(cómputo simbólico), J y K (análisis financiero). Los lenguajes de uso
específico usados comúnmente como SQL y Lex/Yacc, utilizan algunos
elementos de programación funcional, especialmente al procesar valores
mutables. Las hojas de cálculo también pueden ser consideradas
lenguajes de programación funcional.

\subsubsection*{Programación lógica:}
\label{sec:orgfcb870a}
basado en la definición de relaciones lógicas,
 está representado por Prolog.


\subsubsection*{Programación con restricciones:}
\label{sec:orgc52a58e}
similar a la lógica usando
 ecuaciones. Casi todos los lenguajes son variantes del Prolog.

\subsubsection*{Programación multiparadigma:}
\label{sec:org54ecf70}
es el uso de dos o más paradigmas dentro
  de un programa. El lenguaje Lisp se considera multiparadigma. Al
  igual que Python, que es orientado a objetos, reflexivo, imperativo y
  funcional.1

\subsubsection*{Lenguaje específico del dominio o DSL:}
\label{sec:org1899031}
se denomina así a los
  lenguajes desarrollados para resolver un problema específico,
  pudiendo entrar dentro de cualquier grupo anterior. El más
  representativo sería SQL para el manejo de las bases de datos, de
  tipo declarativo, pero los hay imperativos, como el Logo.




\section*{Programación Funcional}
\label{sec:org0d92b5b}

La programación funcional, o mejor dicho, los lenguajes de
programación funcionales, son aquellos lenguajes donde las variables
no tienen estado — no hay cambios en éstas a lo largo del tiempo — y
son inmutables — no pueden cambiarse los valores a lo largo de la
ejecución. Además los programas se estructuran componiendo expresiones
que se evalúan como funciones. Dentro de los lenguajes funcionales
tenemos Lisp, Scheme, Clojure, Haskell, OCaml y Standard ML, entre
otros.  Estos lenguajes están diversidad de tipificación, donde se
encuentran lenguajes dinámicos, estáticos y estáticos fuertes.


En los lenguajes funcionales las instrucciones cíclicas como for,
while y do-while no existen. Todo se procesa usando recursividad y
funciones de alto orden.

Esto se debe a los fundamentos matemáticos de la mayoría de los
lenguajes funcionales, principalmente con bases en el sistema formal
diseñado por Alonzo Church para definir cómputos y estudiar las
aplicaciones de las funciones llamado Cálculo Lambda. En este sistema
formal se puede expresar recursividad en las funciones, y entre otras
cosas interesantes, se pueden expresar combinadores — funciones sin
variables libres — como el Combinador de Punto Fijo o Y-Combinator,
que expresa recursividad sin hacer llamadas recursivas.

En el Cálculo Lambda existen tres transformaciones esenciales, la
conversión \(α\), la reducción \(β\) y la conversión \(η\). En la conversión
α se sustituyen los nombres de las variables para dar mas claridad a
la aplicación de las funciones, por ejemplo evitando duplicados en sus
nombres. En la reducción \(β\) se traza el llamado de las funciones
sustituyendo las funciones por sus expresiones resultantes.

Finalmente en las conversiones η se busca las equivalencias de trazado
de funciones sustituyéndolas por sus equivalentes. Estas
transformaciones también pueden ser aplicadas en los lenguajes
funcionales — o en su mayoría — dando lugar lenguajes que cuentan con
una gran expresividad y consistencia.


Les pondré el clásico ejemplo del chiste geek del castigo “Debo poner
atención en clases”. La respuesta geek expresada en PHP esta escrita a
continuación.  Donde PHP es un lenguaje dinámico, no necesita declarar
variables y es un lenguaje orientado a objetos con raíces imperativas.
Sus instrucciones son paso a paso, y no constituyen una única
expresión reducible.

\begin{minted}[]{c}
<?php
    /* codigo PHP */
    for ($i = 0; $i < 500; $i++) {
        echo "Debo poner atención en clase";
    }
 ?>
\end{minted}

Si usamos Haskell como ejemplo, que es un lenguaje funcional con
tipificación estática fuerte, requiere que las variables sean
declaradas con un tipo — la mayoría de las veces — y es muy expresivo,
donde el siguiente ejemplo dice repetir la cadena, tomar 500 elementos
y con esa lista ejecutar la función monádica putStrLn, que esta hecha
para el Monad IO e imprime  el mensaje las 500 veces solicitada.


\begin{minted}[]{c}
 module Main (main) where

 -- codigo Haskell

 main :: IO ()
 main = mapM_ putStrLn $ take 500 $ repeat "Debo poner atención"
\end{minted}


En Lisp sería similar, pero Lisp es de tipificación dinámica y no
necesita declarar variables, dando lugar a un programa muy simple de
una sola linea.  Donde también tenemos lenguajes como Clojure, que es
un dialecto de Lisp y soporta construcciones muy similares a las del
ejemplo en Lisp, dando lugar a programas expresivos y simples, pero
que corren sobre la máquina virtual de Java o JVM.


\begin{minted}[]{lisp}
 ;;; codigo Lisp

 (loop repeat 500 do (format t "Debo poner atencion en clases~%"))
\end{minted}


Un ejemplo clásico para la conversión η en Haskell, es reducir las
llamadas a funciones en su combinador de identidad. Por ejemplo se
tiene la función \(f(g(x))\), que en Cálculo Lambda se expresa como
\(λx.(λy.y)x\), se puede reducir a \(g(x)\), que se expresa como \(λy.y\) en
Cálculo Lambda. Esto expresado en Haskell, se vería como el siguiente
ejemplo, donde absN y absN’ son funciones equivalentes y absN’ es la
reducción \(η\) de absN.



\begin{minted}[]{haskell}
 absN :: Num a => a -> a
 absN n = abs n

 absN' :: Num a => a -> a
 absN' = abs
\end{minted}


Actualmente los lenguajes orientados a objetos más comunes están
integrando características funcionales, como Java, que acaba de
incluir funciones anonimas.  Pero también están los lenguajes que a lo
largo de su historia han sido multi-paradigma, como Python, e
implementa características funcionales, procedurales y orientadas a
objetos. El bien conocido algoritmo para verificar si un RUT es válido
o no, se puede expresar funcionalmente en Python como esta escrito en
el siguiente ejemplo.


\begin{minted}[]{python}
 def val_rut(rut):
     """
     Valida un string con un RUT con el guion incluido, retornando
     cero si es valido.

     Ejemplo: print(val_rut("22222222-2"))
     """
     return cmp(rut[-1],
		str((range(10) + ['K'])[
                    (11 - sum(map(lambda x: (int(x[0]) * x[1]),
                                  zip(reversed(rut[:-2]),
                                      (2 * range(2, 8))))) % 11)]))

\end{minted}


Como se aprecia en el ejemplo, la validación se realiza utilizando
 expresiones o llamadas a funciones, sin uso de variables con estado y
 mutabilidad, donde cada llamada a una función se puede reducir a un
 valor determinado, y como resultado final se tiene un valor cero o
 distinto de cero que indica si el RUT es válido.  Este mismo
 algoritmo funcional, se puede expresar en Haskell con llamadas muy
 similares, debido a que los nombres de las funciones y funciones de
 alto orden son bastante comunes entre los lenguajes funcionales.


\begin{minted}[]{haskell}
valRut :: String -> Bool
valRut s = (((['0'..'9'] ++ ['K'])
             !! (11 - sum(zipWith (*)
                          (fmap digitToInt $ drop 2 $ reverse s)
                          (take 10 $ cycle [2..7])) `mod` 11)) == (last s))
\end{minted}


De estos dos ejemplos, se puede decir que son funciones puras,
principalmente debido a que no tienen variables libres y son una única
expresión sin estado y no mutable a lo largo de la ejecución. El
problema de la pureza es conceptualmente algo que se idealiza en la
programación funcional, siendo abordado de diferentes formas por
diferentes lenguajes. El objetivo es mantener las funciones y rutinas
puras. En Haskell, con su abstracción más clásica conocida con el
nombre de Mónada, permite entregar pureza a expresiones que parecen no
ser puras, y en términos muy sencillos el Mónada reúne una identidad y
una composición de funciones del tipo \(f(g(x))\), todo a través de un
tipo de dato que permite componer funciones sin abandonar ese tipo de
dato y darle un aspecto procedural.


\section*{Lisp}
\label{sec:org53678f5}
El Lisp (o LISP) es una familia de lenguajes de programación de
computadora de tipo multiparadigma con una larga historia y una
sintaxis completamente entre paréntesis.

clisp

4 + 5 / 34 + 67 * 2 
4 + (/ 5 34) + (* 67 2)
A= (/ 5 34)
B= (* 67 2)
4 + A + B
(+ 4 A) + B
C = (+ 4 A)
C + B
\begin{itemize}
\item C B
\item 4 5
\end{itemize}

Para cargar en un archivo y ejecutarlo en clisp se utiliza la función load. 

\begin{minted}[]{lisp}
;(load "path")
(load "/home/likcos/Materias/Prolog/public/src/ejemplo1.lisp")
\end{minted}

\begin{minted}[]{lisp}
(defun hola()
(print "Hola mundo Prolog" )
)
(hola ')
\end{minted}




\begin{minted}[]{lisp}
(defun suma(a b)
  (+ a b)
)
(suma 3 4)

\end{minted}

Resolver 

\begin{enumerate}
\item 8, 3, -2,-7,-12,
\item 3,6,12,24,48,
\item 4,9,16,25,36,49,
\item 5,10,17,26,37,50,
\item 6,11,18,27,38,51,
\item 3,8,15,24,35,48,
\item -4,9,-16,25,-36,49,
\item 4,-9,16,-25,36,-49,
\item \(\cfrac{2}{4},\cfrac{5}{9},\cfrac{8}{16},\cfrac{11}{25},\cfrac{14}{36}\)
\item \(-5,\cfrac{7}{2},-\cfrac{9}{3},\cfrac{11}{4},-\cfrac{13}{5}\)
\end{enumerate}


\begin{minted}[]{lisp}
(defparameter *pi* 3.1415)

(defun areaCirculo(r)
(* *pi* (* r r)  )
)

'(+ 5 6 )
eval
\end{minted}


\subsection*{Expresiones}
\label{sec:orgd82c906}

Es particularmente cierto que la mejor forma de aprender Lisp es
usándolo, porque se trata de un lenguaje interactivo. Cualquier
sistema Lisp, incluye una interfaz interactiva llamada top-level. Uno
escribe expresiones Lisp en el top-level, y el sistema despliega sus
valores. El sistema normalmente despliega un indicador llamado prompt
(>) señalando que está esperando que una expresión sea escrita. Por
ejemplo, si escribímos el entero 1 después del prompt y tecleamos
enter, tenemos:

\begin{minted}[]{lisp}
> 1
1
\end{minted}

el sistema despliega el valor de la expresión, seguida de un nuevo prompt,
indicando que está listo para evaluar una nueva expresión. En este caso, el
sistema desplegó lo mismo que tecleamos porque los números, como otras
constantes, evalúan a si mismos. Las cosas son más interesantes cuando una
expresión

\begin{minted}[]{lisp}
> (+ 2 3)
5
\end{minted}

En la expresión (+ 2 3) el símbolo + es llamado el operador y los números 3 y 4 son sus argumentos (o parámetros actuales, siguendo la notación
introducida en el capítulo anterior). Como el operador viene al principio de
la expresión, esta notación se conoce como prefija y aunque parezca extraña,
veremos que es muy práctica. Por ejemplo, si queremos sumar tres números
en notación infija, necesitaríamos usar dos veces el operador +: 2+3+4. En
Lisp, las siguientes sumas son válidas:

\begin{minted}[]{lisp}
(+)
0
(+ 2)
2
(+ 2 3)
5
(+ 2 3 5)
10
\end{minted}

Como los operadores pueden tomar un número variable de argumentos,
es necesario utilizar los paréntesis para indicar donde inicia y donde termina
una expresión. Las expresiones pueden anidarse, esto es, el argumento de
una expresión puede ser otra expresión compleja. Ej.

\begin{minted}[]{lisp}
(/ (- 7 1)(- 4 2))
3
\end{minted}

En español esto corresponde a siete menos uno, dividido por cuatro
menos dos.  Estética minimalista, esto es todo lo que hay que decir
sobre la notación en Lisp. Toda expresión Lisp es un átomo, como 1, o
bien es una lista que consiste de cero o más expresiones delimitadas
por paréntesis. Como veremos, código y datos usan la misma notación en
Lisp.

\subsection*{Evaluación}
\label{sec:org2318d71}

Veamos más en detalle como las expresiones son evaluadas para
desplegar su valor en el top-level. En Lisp, + es una función y (+
2 3) es una llamada a la función. Cuando Lisp evalúa una llamada a
alguna función, lo hace en dos pasos:

\begin{itemize}
\item Los argumentos de la llamada son evaluados de izquierda a derecha.
En este caso, los valores de los argumentos son 2 y 3,
respectivamente.

\item Los valores de los argumentos son pasados a la función nombrada por
el operador. En este caso la función + que regresa 5.
\end{itemize}

Si alguno de los argumentos es a su vez una llamada de función, será
evaluado con las mismas reglas. Ej. Al evaluar la expresión (/ (- 7 1) (- 4 2))
pasa lo siguiente.

\begin{enumerate}
\item Lisp evalúa el primer argumento de izquierda a derecha (- 7 1). 7 es
evaluado como 7 y 1 como 1. Estos valores son pasados a la función -
que regresa 6.
\item El siguiente argumento (- 4 2) es evaluado. 4 es evaluado como 4 y 2
como 2. Estos valores son pasados a la función - que regresa 2.
\item Los valores 6 y 2 son pasados a la función / que regresa 3.
\end{enumerate}


No todos los operadores en Lisp son funciones, pero la mayoría lo son.
Todas las llamadas a función son evaluadas de esta forma, que se conoce
como regla de evaluación de Lisp. Los operadores que no siguen la regla de
evaluación se conocen como operadores especiales. Uno de estos operadores
especiales es quote (’). La regla de evaluación de quote es –No evalues nada,
despliega lo que el usuario tecleo, verbatim. 

\begin{minted}[]{lisp}
(quote (+ 2 3))
(+ 2 3)
'( + 2 3)
(+ 2 3)
\end{minted}

Lisp provee el operador quote como una forma de evitar que una
expresión sea evaluada. En la siguiente sección veremos porque esta
protección puede ser útil.

\subsection*{Datos}
\label{sec:orgd64f70c}

Lisp ofrece los tipos de datos que podemos encontrar en otros
lenguajes de programación, y otros que no. Ya hemos usado enteros en
los ejemplos precedentes. Lascadenas de caracteres se delimita por
comillas, por ejemplo, “Hola mundo”. Enteros y cadenas evalúan a ellos
mismos, como las constantes.  Dos tipos de datos propios de Lisp son
los símbolos y las listas. Los símbolos son palabras. Normalmente se
evaluan como si estuvieran escritos en mayúsculas, independientemente
de como fueron tecleados:

\begin{minted}[]{lisp}
'uno
uno
\end{minted}

Los símbolos por lo general no evaluan a si mismos, así que si es
necesario referirse a ellos, se debe usar quote, como en ejemplo
anterior, de lo contrario, se producirá un error ya que el símbolo uno
no está acotado (no tiene ligado ningún valor en este momento).

Las listas se representan como cero o más elementos entre paréntesis. Los
elementos pueden ser de cualquier tipo, incluidas las listas. Se debe usar
quote con las listas, ya que de otra forma Lisp las tomaría como una llamada
a función. Veamos algunos ejemplos:

\begin{minted}[]{lisp}
’(Mis 2 "ciudades")
(MIS 2 "CIUDADES")
’(La lista (a b c) tiene 3 elementos)
(LA LISTA (A B C) TIENE 3 ELEMENTOS)
\end{minted}

Observen que quote protege a toda la expresión, incluidas las
sub-expresiones en ella. La lista (a b c), tampoco fue
evaluada. También es posible construir listas usando la función list:

\begin{minted}[]{lisp}
(list ’mis (+ 4 2) "colegas")
(MIS 6 COLEGAS)
\end{minted}

Estética minimalista y pragmática, observen que los programas Lisp se
representan como listas. Si el argumento estético no bastará para defender la
notación de Lisp, esto debe bastar –Un programa Lisp puede generar código
Lisp! Por eso es necesario quote. Si una lista es precedida por el operador
quote, la evaluación regresa la misma lista, en otro caso, la lista es evaluada
como si fuese código. Por ejemplo:

\begin{minted}[]{lisp}
(list ’(+ 2 3) (+ 2 3))
((+ 2 3) 5)
\end{minted}


\subsection*{Funciones en Lisp}
\label{sec:orga21e3bf}
Es posible definir nuevas funciones con defun que toma normalmente tres
argumentos: un nombre, una lista de parámetros y una o más expresiones
que conforman el cuerpo de la función. Ej. Así definiríamos tercero:

\begin{minted}[]{lisp}
(defun tercero(lst)
  (caddr lst)
)
\end{minted}

El primer argumento de defun indica que el nombre de nuestra función
definida será tercero. El segundo argumento (lst) indica que la
función tiene un sólo argumento, lst. Un símbolo usado de esta forma
se conoce como variable. Cuando la variable representa el argumento de
una función, se conoce como parámetro. El resto de la definición
indica lo que se debe hacer para calcular el valor de la función, en
este caso, para cualquier lst, se calculará el primer elemento, del
resto, del resto del parámetro (caddr lst).

\begin{minted}[]{lisp}
(tercero '(a b c d e))
C
\end{minted}

Ahora que hemos introducido el concepto de variable, es más sencillo
entender lo que es un símbolo. Los símbolos son nombres de variables, que
existen con derechos propios en el lenguaje Lisp. Por ello símbolos y listas
deben protegerse con quote para ser accesados. Una lista debe protegerse
porque de otra forma es procesada como si fuese código; un símbolo debe
protegerse porque de otra forma es procesado como si fuese una variable.
Podríamos decir que la definición de una función corresponde a la versión
generalizada de una expresión Lisp. Ej. La siguiente expresión verifica si la
suma de 1 y 4 es mayor que 3:

\begin{minted}[]{lisp}
( >(+ 1 4) 3)
\end{minted}

Substituyendo los números partículares por variables, podemos definir
una función que verifica si la suma de sus dos primeros argumentos es
mayor que el tercero:

\begin{minted}[]{lisp}
(defun suma-mayor-que (x  y z)
  (> (+ x y) z)
)

(suma-mayor-que 1 4 3)
\end{minted}

Lisp no distigue entre programa, procedimiento y función; todos cuentan
como funciones y de hecho, casi todo el lenguaje está compuesto de funciones. Si se desea considerar una función en partícular como main, es posible
hacerlo, pero cualquier función puede ser llamada desde el top-level. Entre
otras cosas, esto significa que posible probar nuestros programas, pieza por
pieza, conforme los vamos escribiendo, lo que se conoce como programación
incremental (bottom-up).

\subsection*{Funciones CAR y CDR}
\label{sec:org9cac572}


CAR admite un único argumento que debe ser una lista o una expresión
cuyo valor sea una lista y devuelve el primer elemento de dicha
lista. Como LISP siempre interpreta una lista como una llamada a una
función, necesitamos una manera de pasar una lista a CAR sin que LISP
trate de procesarla como llamada a función.

CDR recibe también una lista y devuelve el resto de la lista después
de eliminar el primer elemento (CAR lista). Por lo tanto puede
contemplarse como la función complementaria de CAR. Una manera de
interpretar una lista en LISP es como la conjunción de un CAR y un
CDR. Se podrá acceder a cualquier término de una lista mediante los
anidamientos apropiados de CAR y CDR.  

\begin{itemize}
\item (CAR lista) ;primer término
\item (CAR (CDR lista)) ;segundo término
\item (CAR (CDR (CDR lista))) ; tercer término
\end{itemize}

Para simplificar las expresiones se permite crear 
nombres compuestos
para anidamientos de CAR y CDR de la siguiente manera:
\begin{itemize}
\item comenzando por una primera letra C
\item una letra A por cada CAR o una D por cada CDR
\item terminando con una letra R
\item así (CAR (CDR (CDR lista))) sería lo mismo que (CADDR lista).
\end{itemize}

De esta manera podemos construir hasta 28 funciones distintas para acceso a
listas. Según Johnson estas 28 concatenaciones se pueden dividir en
cuatro grupos, tomando como base la profundidad a que pueden acceder
en listas anidadas en varios niveles.


\begin{minted}[]{lisp}
;(eval (car '((+ 1 2) 2 3 4 5)))
(cadr '(1 2 3 4 5))
;(cddd  '((1 2 3 4 5) (6 7 8 9 10) (11 12 13 14 15)))

\end{minted}


\begin{minted}[]{lisp}
(defun)


\end{minted}


\begin{minted}[]{lisp}
(defun mylength (list)
(if list 
	(1+ (mylength (cdr list)))
	0))

(defun rec(list)
(print (car list))
(if list 
(rec ( cdr list))
)
)
(defvar a 4)
(format T "Hola mundo ~ d" a)
(rec '((1 2 3 4 4)))
;(mylength '(1 2 3 4))

\end{minted}


\subsection*{Valores de Verdad IF}
\label{sec:org642e34c}

En Lisp, el símbolo t es la representación por default para
verdadero. La representación por default de falso es nil . Ambos
evaluan a si mismos. Ej.  La función listp regresa verdadero si su
argumento es una lista:

\begin{minted}[]{lisp}
(listp '(a b c))
T
(listp 34)
Nil
\end{minted}

Una función cuyo valor de regreseo se intérpreta como un valor de ver-
dad (verdadero o falso) se conoce como predicado. En lisp es común que
el símbolo de un predicado termine en p .  Como nil juega dos roles en
\textbf{Lisp}, las funciones \textbf{null} (lista vacía) y not (negación) hacen
exactamente lo mismo:


\begin{minted}[]{lisp}
(null nil)
T
(not nil)
T
\end{minted}

El condicional (estructura de control) más simple en \textbf{Lisp} es \textbf{if}
. Normalmente toma tres argumentos: una expresión test, una expresión
then y una expresión \textbf{else}. La expresión test es evaluada, si su
valor es verdadero, la expresión \textbf{then} es evaluada; si su valor es
falso, la expresión \textbf{else} es evaluada.

\begin{minted}[]{lisp}
(if (listp '(a b c d ))
	(+ 1 2)
	(+ 3 4)
)
\end{minted}

\begin{minted}[]{lisp}
(if (listp 23)
	(+ 1 2)
	(+ 3 4)
)
\end{minted}

Como \textbf{quote}, \textbf{if*} es un operador especial. No puede implementarse como
una función, porque los argumentos de una función siempre se evaluan,
y la idea al usar if es que sólo uno de sus argumentos sea evaluado.
Si bien el default para representar verdadero es \textbf{t} , todo excepto nil
cuenta como verdadero en un contexto lógico. Ej.

\begin{minted}[]{lisp}
(if 27 1 2)
\end{minted}

\begin{minted}[]{lisp}
(if nil 1 2)
\end{minted}

Dado que solo se evalúa una expresión dentro de un \textbf{if}, es imposible
hacer dos o más cosas separadas dentro de una opción. Sin embargo,
para los casos en los que realmente se desea hacer más de una cosa, se
puede utilizar un comando especial como el \textbf{progn} para insertar
comandos adicionales en una sola expresión. Con \textbf{progn}, solo se
devuelve la última evaluación como el valor de la expresión completa.



\begin{minted}[]{lisp}
(defvar *number-was-odd* nil)
(if (oddp 5)
	(progn (setf *number-was-odd* t)
	   'odd-number)
	'even-number)
\end{minted}

\subsection*{When unless}
\label{sec:org4a5703f}

De igual forma \textbf{Lisp} tiene otros comandos que tienen un \textbf{progn}
implícito los más básicos son \textbf{when} y \textbf{unless}  

\begin{minted}[]{lisp}
(defvar *number-is-odd* nil)

(when (oddp 5)
  (setf *number-is-odd* t)
  'odd-number)

ODD-NUMBER

*number-is-odd*
T

(unless (oddp 4)
  (setf *number-is-odd* nil)
  'even-number)

EVEN-NUMBER

*number-is-odd*
NIL
\end{minted}

Con \textbf{when}, todas las expresiones encerradas se evalúan cuando la
condición es verdad, con \textbf{unless} todas las expresiones encerradas se
evalúan cuando la condición es falsa. La compensación es que estos
comandos no pueden hacer nada cuando la condición se evalúa de manera
opuesta; simplemente devuelven cero y no hacen nada.


\subsection*{Operadores Lógicos and or}
\label{sec:org35a7bad}

Los operadores lógicos \textbf{and} y \textbf{or} parecen condicionales. Ambos toman
cualquier número de argumentos, pero solo evaluan los necesarios
para decidir que valor regresar. Si todos los argumentos son
verdaderos (diferentes de nil ), entonces and regresa el valor del
último argumento. Ej.


\begin{minted}[]{lisp}
(defun ejemplosif(x)
(unless (<= 2 x 24)
	(print "Hola")
	(print "hola2")
)
)

(ejemplosif 21 )

\end{minted}


(setq var (getreal "\nEnter var: "))
(cond
  ((and (>= var 2) (<= var 24 ))(princ "\nvar is between 2 \& 24"))
  ((and (>= var 26) (<= var 48))(princ "\nvar is between 26 \& 48"))
  ((and (>= var 50) (<= var 96))(princ "\nvar is between 50 \& 96"))
  (T nil)
  )

also like this

(cond
  ((<= 2 var 24)(DO THIS))
  ((<= 26 var 48)(DO THIS))
  ((<= 50 var 96)(DO THIS))
  (T nil)
  )


\begin{minted}[]{lisp}
(defvar *arch-enemy* nil)

(defun pudding-eater (person)
  (cond 
	((eq person 'henry) 
	 '(curse you lisp alien – you ate my pudding))
	((eq person 'johnny) 
	 '(i hope you choked on my pudding johnny))
	(t
	 '(why you eat my pudding stranger ?))
	)
)

(pudding-eater 'lalo)

\end{minted}

(espada pistol  )
(espada) (pistola) 
\begin{minted}[]{lisp}

(defparameter *nodo* '((Zerg
						(Terreste
						 (Larva
						  (Unidad que se crea automáticamente en la colmenta))
						 (Zergling
						  (Unidad de ataque cuerpo a cuerpo básica))
						 (Zángano
						  (Unidad no ofensiva cuya tarea consiste en la recolección de mineral y Gas Vespeno)) 
						 (Corruptor
						  (Unidad de ataque bacteriológicos))
						 (Plaga Terran
							(Solo se crean en centros de mandos terran previamente infestados por una Reina Zerg))
						 (Merodeador
						  (Es una mutación del Hidralisco que solo atacan unidades terrestres tras enterrarse en el suelo))
						 (Ultralisco
						  (El ultralisco es la unidad más poderosa de la raza Zerg.)))
						(Aereo
						 (Superamo
						  (Unidad de transporte Zerg))
						 (Mutalisco
						  (Unidad rápida pero de corto rango de ataque y bajo daño))
						 (Atormentador
						  (Unidad pequeña y de rápida producción)))
						)
					   (Terran
						(Terrestres
						 (Soldado Terran
							(Los soldados terran son la primera linea de defensa de los planetas terran))
						 (Tanque de Asedio
							(Unidad de asalto y defensa mecanizada de ataque terrestre))
						 (Goliat
						  (Unidad de apoyo ligera y versatil))
						 (Buitre
						  (Unidad ligera y con la habilidad de poner tres minas arañas)))
						)
					   (protoss)))

(car *nodo*)

\end{minted}


\begin{minted}[]{lisp}
(defparameter *edges* '((living-room (garden west door)
						 (attic upstairs ladder))
						(garden (living-room east door))
						(attic (living-room downstairs ladder))))

(car edges)

\end{minted}


\begin{minted}[]{lisp}
(defparameter *nodos* '((zerg
						 (Terrestres
						  ()
						  )
						 (Aereos)
						 )
						(protos)
						(terran)


						))


(defparameter *nodo* '((Zerg
						(Terreste
						 (Larva
						  (Unidad que se crea automáticamente en la colmenta))
						 (Zergling
						  (Unidad de ataque cuerpo a cuerpo básica))
						 (Zángano
						  (Unidad no ofensiva cuya tarea consiste en la recolección de mineral y Gas Vespeno)) 
						 (Corruptor
						  (Unidad de ataque bacteriológicos))
						 (Plaga Terran
							(Solo se crean en centros de mandos terran previamente infestados por una Reina Zerg))
						 (Merodeador
						  (Es una mutación del Hidralisco que solo atacan unidades terrestres tras enterrarse en el suelo))
						 (Ultralisco
						  (El ultralisco es la unidad más poderosa de la raza Zerg.)))
						(Aereo
						 (Superamo
						  (Unidad de transporte Zerg))
						 (Mutalisco
						  (Unidad rápida pero de corto rango de ataque y bajo daño))
						 (Atormentador
						  (Unidad pequeña y de rápida producción)))
						)
					   (Terran
						(Terrestre
						  (Soldado Terran
							(Los soldados terran son la primera linea de defensa de los planetas terran))
						 (Murcielago de Fuego
							(Unidad equipada con el traje de combate pesado CMC660))
						 (Medico
						  (No posse ningun tipo de ataque físico))
						 (Fantasma
						  (Inteligencia y espionaje))
						 )
						(Aereo
						 (Espectro
						  (Unidad Aérea básica))
						 (Crucero de batalla
						  (Poderosa unidad de combante))
						 (Nave de la ciencia
						  (Unidad aérea de apoyo y especialista con la facultad de detectar unidades enemigas enterradas o invisible))
						 )
						)
					   (protoss
						(Terrestres
						 (Sonda
						  (Recolectan el gas vespeno y el mineral))
						 (Fanático
						  (Son la columna vertebral del ejército Protoss))
						 (Dragoon
						  (Son parecidos a arañas mecánicas))
						 )
						(Aereo
						 (Observador
						  (No tiene armamento))
						 (Lanzadera
						  (Sirve para transportar unidades))))
					   ))

;(caar *nodo*)
(format t "Estas pensando en los   ~a" (car (assoc 'protoss *nodo*)))
;(setq aux (car (assoc 'protoss *nodo*)))
;(print aux)
(assoc 'Aereo (cdr (assoc 'Terran *nodo*)))
;(assoc 'Terrestres (cdr (assoc 'protoss *nodo*)))
\end{minted}
Hola  PROTOSS
Hola  PROTOSS


\begin{minted}[]{lisp}
(defparameter *lista*  '(Terran Zerg Protoss ))

(defun recorre(lista)
(format t "Tu personaje es ~a ~%" (car lista ))
(setq aux (read))
(terpri)
(if (= aux 1 )

(if lista
(recorre (cdr lista))))

(recorre *lista*)
;(print (car *lista*))
;(print (cdr *lista*))
\end{minted}
NIL
(ZERG PROTOSS)
TERRAN
(TERRAN ZERG PROTOSS)

\subsection*{Funciones lambda en Lisp}
\label{sec:org962f973}

En el ámbito de la programación, una expresión lambda, también
denominada función lambda, función literal o función anónima, es una
subrutina definida que no está enlazada a un identificador. Las
expresiones lambda a menudo son argumentos que se pasan a funciones de
orden superior, o se usan para construir el resultado de una función
de orden superior que necesita devolver una función.​ Si la función
solo se usa una vez o un número limitado de veces, una expresión
lambda puede ser sintácticamente más simple que usar una función con
nombre. Las funciones lambda son muy comunes en los lenguajes de
programación funcional y en otros lenguajes con funciones de primera
clase, en los que cumplen el mismo papel para el tipo de función que
los literales para otros tipos de datos.


\subsubsection*{Funciones como valores de primera clase}
\label{sec:org7ce077c}

\begin{itemize}
\item Sintaxis
\label{sec:org1fb9e4e}
\begin{itemize}
\item (nombre de la función) ; recupera la función objeto de ese nombre
\item \#'nombre ; azúcar sintáctica para (nombre de la función)
\item (símbolo de función de símbolo); devuelve la función vinculada al símbolo
\item (función funcs args \ldots{}); llamar a la función con args
\item (aplicar la función arglista); Función de llamada con argumentos dados en una lista.
\item (aplicar la función arg1 arg2 \ldots{} argn arglist); Función de llamada con argumentos dados por arg1, arg2, \ldots{}, argn y el resto en la lista arglist
\end{itemize}


\item Parámetros
\label{sec:org280af3d}

\begin{center}
\begin{tabular}{ll}
Parámetro & Detalles\\
\hline
Nombres & Algún símbolo (no evaluado que nombra una función)\\
\hline
Símbolo & Un símbolo\\
\hline
Función & Una función que se va llamar\\
\hline
arg & cero o más argumentos (no una lista de argumentos)\\
\hline
arglista & Una lista que contiene argumentos para pasar a una función\\
\hline
arg1,arg2,..argn & \\
\hline
\end{tabular}
\end{center}


\item Observaciones
\label{sec:orga5604ad}

Cuando se habla de lenguajes similares a Lisp, existe una distinción
común entre lo que se conoce como Lisp-1 y Lisp-2. En un Lisp-1, los
símbolos solo tienen un valor y si un símbolo se refiere a una
función, entonces el valor de ese símbolo será esa función. En un
Lisp-2, los símbolos pueden tener valores y funciones asociadas por
separado, por lo que se requiere una forma especial para referirse a
la función almacenada en un símbolo en lugar del valor.

Common Lisp es básicamente un Lisp-2, sin embargo, en realidad hay más
de 2 espacios de nombres (cosas a las que los símbolos pueden
referirse): los símbolos pueden referirse a valores, funciones, tipos
y etiquetas, por ejemplo.


\begin{itemize}
\item Definiendo funciones anónimas.
\label{sec:orga006ec3}
Las funciones en Common Lisp son valores de primera clase . Se puede
crear una función anónima usando lambda . Por ejemplo, aquí hay una
función de 3 argumentos que luego llamamos usando funcall

\begin{minted}[]{lisp}
(lambda (a b c) (+ a (* b c)))
(LAMBDA (A B C)) {10034F484B}>
(defvar *foo* (lambda (a b c) (+ a (* b c))))
*FOO*
(funcall *foo* 1 2 3)
7
\end{minted}

\textbf{Funcall}: La función funcall llama a la función proporcionada con argumentos específicos 

Las funciones anónimas también se pueden utilizar directamente. Common
Lisp proporciona una sintaxis para ello.

\begin{minted}[]{lisp}
((lambda (a b c) (+ a (* b c)))    ; Funcion lambda 
                                   ; 
  1 2 3)                           ; Argumentos

\end{minted}


Las funciones anónimas también se pueden almacenar como funciones globales:
\begin{minted}[]{lisp}
(let ((a-function (lambda (a b c) (+ a (* b c)))))      ; funcion anonima 
  (setf (symbol-function 'some-function) a-function))   ; se asigna a 'some-function

(some-function 1 2 3)                                   ; se llama con el mismo nombre some-funtion

\end{minted}


Las expresiones lambda citadas no son funciones. Tenga en cuenta que
las expresiones lambda citadas no son funciones en Common Lisp. Esto
no funciona:

\begin{minted}[]{lisp}
(funcall '(lambda (x) x)
         42)

\end{minted}

Para convertir una expresión lambda entrecomillada a una función, use
coerce , eval o funcall :

\begin{minted}[]{lisp}
(coerce '(lambda (x) x) 'function)
#<anonymous interpreted function 4060000A7C>

(eval '(lambda (x) x))
#<anonymous interpreted function 4060000B9C>

(compile nil '(lambda (x) x))
#<Function 17 4060000CCC>

\end{minted}

Haciendo referencia a las funciones existentes Cualquier símbolo en
Common Lisp tiene una ranura para una variable a vincular y una ranura
separada para una función a vincular.

Se tiene que tener  en cuenta que la denominación en este ejemplo es solo para
ilustración. Las variables globales no deberían llamarse foo , sino
\textbf{foo} . La última notación es una convención para dejar en claro que
la variable es una variable especial que utiliza el enlace dinámico .
\begin{minted}[]{lisp}
(boundp 'foo) ;is FOO defined as a variable?
NIL
(defvar foo 7)
FOO
(boundp 'foo)
T
foo
7
(symbol-value 'foo)
7
(fboundp 'foo) ;is FOO defined as a function?
NIL
 (defun foo (x y) (+ (* x x) (* y y)))
FOO
 (fboundp 'foo)
T
 foo
7
(symbol-function 'foo)
#<FUNCTION FOO>
(function foo)
#<FUNCTION FOO>
(equalp (quote #'foo) (quote (function foo)))
T
(eq (symbol-function 'foo) #'foo)
T
(foo 4 3)
25
(funcall foo 4 3)
;get an error: 7 is not a function
(funcall #'foo 4 3)
25
(defvar bar #'foo)
BAR
bar
#<FUNCTION FOO>
(funcall bar 4 3)
25
#'+
#<FUNCTION +>
(funcall #'+ 2 3)
5
\end{minted}
\end{itemize}


\item Funciones de orden superior
\label{sec:orgf44d405}
Common Lisp contiene muchas funciones de orden superior que son
funciones pasadas para los argumentos y las llaman. Tal vez los más
fundamentales sean funcall y de apply :

\begin{minted}[]{lisp}
(list 1 2 3)
(1 2 3)
(funcall #'list 1 2 3)
(1 2 3)
(funcall #'list 1 2 3 4 5)
(1 2 3 4 5)
(apply #'list '(1 2 3))
(1 2 3)
(apply #'list 1 2 '(4 5))
(1 2 3 4 5)
(apply #'+ 1 (list 2 3))
6
(defun my-funcall (function &rest args)
           (apply function args))
MY-FUNCALL
(my-funcall #'list 1 2 3)
(1 2 3)
\end{minted}

Hay muchas otras funciones de orden superior que, por ejemplo, aplican
una función muchas veces a los elementos de una lista.

\begin{minted}[]{lisp}
(map 'list #'/ '(1 2 3 4))
(1 1/2 1/3 1/4)
(map 'vector #'+ '(1 2 3 4 5) #(5 4 3 2 10))
#(6 6 6 6 15)
(reduce #'+ '(1 2 3 4 5))
15
(remove-if #'evenp '(1 2 3 4 5))
(1 3 5)

\end{minted}

Sumando una lista 

La función de reducción se puede utilizar para sumar
los elementos de una lista.

\begin{minted}[]{lisp}
(reduce '+ '(1 2 3 4))
;;=> 10
\end{minted}

Por defecto, reducir realiza una reducción asociativa a la izquierda ,
lo que significa que la suma 10 se calcula como

\begin{minted}[]{lisp}
(+ (+ (+ 1 2) 3) 4)
\end{minted}

Los primeros dos elementos se suman primero, y luego ese resultado (3)
se agrega al siguiente elemento (3) para producir 6, que a su vez se
agrega a 4, para producir el resultado final.

Esto es más seguro que usar aplicar (por ejemplo, en (aplicar '+' (1 2
3 4)) porque la longitud de la lista de argumentos que se puede pasar
a aplicar es limitada (consulte el límite de argumentos de llamadas ),
y la reducción funcionará Con funciones que solo toman dos argumentos.

Al especificar el argumento de la palabra clave desde el final ,
reducir procesará la lista en la otra dirección, lo que significa que
la suma se computa en el orden inverso. Es decir


\begin{minted}[]{lisp}
(reduce '+ (1 2 3 4) :from-end t)
;;=> 10
\end{minted}

está computando

\begin{minted}[]{lisp}
(+ 1 (+ 2 (+ 3 4)))
\end{minted}


Implementación de reversa y reventa.  Common Lisp ya tiene una función
inversa , pero si no lo tenía, entonces podría implementarse
fácilmente usando reducir . Dada una lista como

\begin{minted}[]{lisp}
(1 2 3) === (cons 1 (cons 2 (cons 3 '())))
\end{minted}

la lista invertida es

\begin{minted}[]{lisp}
(cons 3 (cons 2 (cons 1 '()))) === (3 2 1)
\end{minted}

Puede que no sea un uso obvio de reducir , pero si tenemos una función
de "contras invertida", digamos xcons , de modo que

\begin{minted}[]{lisp}
(xcons 1 2) === (2 . 1)
\end{minted}

Entonces

\begin{minted}[]{lisp}
(xcons (xcons (xcons () 1) 2) 3)
\end{minted}

Que es una reducción.

\begin{minted}[]{lisp}
(reduce (lambda (x y)
          (cons y x))
        '(1 2 3 4)
        :initial-value '())
;=> (4 3 2 1)
\end{minted}


Common Lisp tiene otra función útil, revappend , que es una
combinación de reversa y anexa . Conceptualmente, invierte una lista y
la agrega a alguna cola:

\begin{minted}[]{lisp}
(revappend '(3 2 1) '(4 5 6))
;;=> (1 2 3 4 5 6)

\end{minted}

Esto también se puede implementar con reducir . De hecho, es lo mismo
que la implementación del reverso anterior, excepto que el valor
inicial debería ser (4 5 6) en lugar de la lista vacía.

\begin{minted}[]{lisp}
(reduce (lambda (x y)
          (cons y x))
        '(3 2 1)
        :initial-value '(4 5 6))
;=> (1 2 3 4 5 6)
\end{minted}

Cierres

Las funciones recuerdan el ámbito léxico en el que se
definieron. Debido a esto, podemos incluir un lambda en un let para
definir los cierres.

\begin{minted}[]{lisp}
(defvar *counter* (let ((count 0))
                    (lambda () (incf count))))

(funcall *counter*) ;; => 1
(funcall *counter*) ;; = 2

\end{minted}

En el ejemplo anterior, la variable de contador solo es accesible a la
función anónima. Esto se ve más claramente en el siguiente ejemplo.

\begin{minted}[]{lisp}
(defvar *counter-1* (make-counter))
(defvar *counter-2* (make-counter))

(funcall *counter-1*) ;; => 1
(funcall *counter-1*) ;; => 2
(funcall *counter-2*) ;; => 1
(funcall *counter-1*) ;; => 3

\end{minted}
Definiendo funciones que toman funciones y devuelven funciones.
Un ejemplo simple:

\begin{minted}[]{lisp}
(defun make-apply-twice (fun)
           "return a new function that applies twice the function`fun' to its argument"
           (lambda (x)
             (funcall fun (funcall fun x))))

(funcall (make-apply-twice #'1+) 3)

;(let ((pow4 (make-apply-twice (lambda (x) (* x x)))))
;           (funcall pow4 3))

\end{minted}

El ejemplo clásico de la composición de la función : ( f ∘ g ∘ h ) ( x ) = f ( g ( h ( x )):

\begin{minted}[]{lisp}
(defun compose (&rest funs)
           "return a new function obtained by the functional compositions of the parameters"
           (if (null funs) 
               #'identity
               (let ((rest-funs (apply #'compose (rest funs))))
                 (lambda (x) (funcall (first funs) (funcall rest-funs x))))))

(defun square (x) (* x x))

(funcall (compose #'square #'1+ #'square) 3)
100  ;; => equivalent to (square (1+ (square 3)))

\end{minted}


\begin{minted}[]{lisp}
( lambda (n)
  ((lambda (fact) (fact fact 1 n))
   (lambda (f P n) (if (<= n 1) P (f f (* n P) (- n 1))))))


\end{minted}

\begin{minted}[]{lisp}
((lambda (x) (* x x)) 3)
\end{minted}
\end{itemize}


\subsection*{Ejercicios}
\label{sec:orgae307f1}

(defpackage "EJERCICIOS")

;;1.1 Definir una función que calcule el valor de:
;; F = 1/ sqrt (a*2 + b*2 + c)


;;1.2 Definir una función que devuelva la longitud de una
;;circunferencia, dando como parámetro el radio R de la misma


;;1.3 Definir una función que pase de grados centígrados a grados
;;Fahrenheit, sabiendo que: F = (C + 40) x 1.8 - 40

;;1.4 Definir una función que, dados tres argumentos numéricos,
;;devuelva cuál es el mediano, utilizando MAX y MIN.

;;1.5 Definir un predicado que dados A, B y C como argumentos devuelva
;;    T si B2 - 4AC es menor que cero.

;;


\section*{Prolog}
\label{sec:org07743b7}

Una regla sirve para representar conocimiento que en lenguaje natural
 se expresa mediante una sentencia condicional. Por ejemplo, si en
 lenguaje natural decimos «si X es padre de Y entonces Y es hijo de X»
 , en Prolog escribiremos: hijo(Y,X) :- padre(X,Y).  El símbolo «:-»
 significa «si» , y la traducción directa al lenguaje natural de la
 regla es: «Y es hijo de X si X es padre de Y» .  Las reglas resultan
 muy útiles para definir nuevos predicados a partir de otros
 previamente definidos. Por ejemplo, podríamos tener un conjunto de
 hechos de la forma «padre(juan, luis)» , «padre(luis,jaime)» , etc.,
 que definen el predicado (o relación) «padre» por extensión. La regla
 anterior define intensionalmente la relación «hijo» .  Muchas
 expresiones que en lenguaje natural no tienen explícitamente la forma
 condicional pueden representarse de este modo manteniendo su
 significado: «todos los hombres son mortales» es equivalente a decir
 «si X es un hombre entonces X es mortal» ; en Prolog:

\begin{minted}[]{prolog}
mortal(X) :- hombre(X).
\end{minted}

En general, una regla tiene una «cabeza» y un «cuerpo» . La cabeza es
un predicado, y el cuerpo una conjunción de literales; para indicar la
conjunción se utiliza una coma separando a los predicados del cuerpo:
una definición de «abuelo» es:
\begin{minted}[]{prolog}
 abuelo(X,Y) :- padre(X,Z),padre(Z,Y).
\end{minted}

(X es abuelo de Y si X es padre de algún individuo Z que, a su vez, es
padre de Y).  Pero esta definición estaría incompleta: sólo cubre los
abuelos paternos. Podemos completarla añadiendo otra regla:

\begin{minted}[]{prolog}
abuelo(X,Y) :- padre(X,Z),madre(Z,Y).
\end{minted}

Escribir dos o más reglas para definir un predicado es la manera
normal de expresar en Prolog lo que en lógica sería una disyunción. En
este caso, «X es abuelo de Y si\ldots{} o bien si\ldots{}» . También puede
expresarse explícitamente la disyunción mediante «;» :

\begin{minted}[]{prolog}
abuelo(X,Y) :- padre(X,Z),(padre(Z,Y);madre(Z,Y)).
\end{minted}

pero normalmente se prefiere la versión en dos reglas por su mayor
claridad.  También se puede definir introduciendo un concepto
intermedio, «progenitor» (padre o madre):

\begin{minted}[]{prolog}
 progenitor(X,Y) :- padre(X,Y).  
 progenitor(X,Y) :- madre(X,Y).  
 abuelo(X,Y) :- padre(X,Z),progenitor(Z,Y).
\end{minted}

El cuerpo de la regla puede contener literales negativos. Por ejemplo:

\begin{minted}[]{prolog}
hermano(X,Y) :- progenitor(Z,X),  
                 progenitor(Z,Y), not (X=Y).
\end{minted}


importante: no debe confundirse «reglas de inferencia» con «reglas»
(de Prolog) (y tampoco con «reglas gramaticales»). Las «reglas» de
Prolog son sentencias condicionales que se satisfacen para unas
interpretaciones pero no para otras. Una regla de inferencia es un
esquema general de razonamiento que, formalizado, se representa por
una sentencia válida (Apartado . Por ejemplo, la regla de Prolog
«men(x) :- pol(x)» corresponde a la sentencia condicional ( A
x)(pol(x) ==>men(x)) se satisface para unas interpretaciones pero no
para otras. Una particularización de la regla de inferencia modus
ponens es: P1: ( A x)(pol(x) ==>men(x)) P2: pol(x) C: men(x) a la que
corresponde la sentencia ( A x)((pol(x) ==>men(x)) /$\backslash$ pol(x)
==>men(x)), que es una sentencia válida (se satisface con cualquier
interpretación de los predicados y cualquier asignación de x).

\subsection*{Corte}
\label{sec:orga5e8ec5}

El corte es un predicado predefinido que no recibe argumentos. Se
representa mediante un signo de admiración (!). Sin duda, es el
predicado más difícil de entender. El corte tiene la espantosa
propiedad de eliminar los puntos de elección del predicado que lo
contiene.

Es decir, cuando se ejecuta el corte, el resultado del objetivo (no
sólo la cláusula en cuestión) queda comprometido al éxito o fallo de
los objetivos que aparecen a continuación. Es como si a Prolog "se le
olvidase" que dicho objetivo puede tener varias soluciones.

Otra forma de ver el efecto del corte es pensar que solamente tiene la
propiedad de detener el backtracking cuando éste se produce. Es decir,
en la ejecución normal el corte no hace nada. Pero cuando el programa
entra en backtracking y los objetivos se recorren marcha atrás, al
llegar al corte el backtracking se detiene repentinamente forzando el
fallo del objetivo.



Para entender de manera simple el uso del corte vamos a comparar dos
predicados que solamente se diferencian en un corte:


\begin{minted}[]{prolog}
 % Sin corte. 
 p(X,Y) :- X > 15, Y > 50. 
 
 p(X,Y) :- X > Y, 

 % Con corte.
 q(X,Y) :- X > 15, !, Y > 50. 
 
 q(X,Y) :- X > Y, 

\end{minted}


Veamos que ocurre si ejecutamos el objetivo p(25,12):

\begin{itemize}
\item Obsérve que ambas cláusulas unifican con la cabeza, luego existen
dos puntos de elección que se anotan.
\item Prolog entra por el primer punto de elección (primera cláusula)
eliminandolo.
\item Prolog ejecuta el primer objetivo del cuerpo (X>15), que tiene
éxito.
\item Prolog ejecuta el segundo objetivo del cuerpo (X>50), que falla.
\item Empieza el backtracking.
\item Se recorren ambos objetivos hacia atrás pero no hay variables que se
hayan ligado en ellos.
\item Encontramos el segundo punto de elección (segunda cláusula) que
detiene el backtracking eliminandolo en el proceso. La ejecución
continúa hacia delante.
\item Prolog ejecuta el cuerpo de la segunda cláusula que consiste en
X>Y. Este objetivo tiene éxito.
\item El objetivo p(25,12) tiene éxito.
\end{itemize}

Ahora comprobamos lo que ocurre cuando éxiste el corte, ejecutamos q(25,12):

\begin{itemize}
\item Ambas cláusulas unifican con la cabeza, luego existen dos puntos de
elección que se anotan.
\item Prolog entra por el primer punto de elección (primera cláusula)
eliminandolo.
\item Prolog ejecuta el primer objetivo del cuerpo (X>15), que tiene
éxito.
\item Se ejecuta el segundo objetivo del cuerpo que es el corte. Por
tanto, se eliminan todos los puntos de elección anotados que son
debidos al objetivo q(25,12). Solamente teníamos uno, que se
elimina.
\item Prolog ejecuta el tercer objetivo del cuerpo (X>50), que falla.
\item Empieza el backtracking.
\item Se recorren ambos objetivos hacia atrás pero no hay variables que se
hayan ligado en ellos.
\item No encontramos ningún punto de elección porque fueron eliminados por
el corte.
\item El objetivo p(25,12) falla.
\end{itemize}

Como puede comprobar, los resultados son sustacialmente diferentes. La
segunda cláusula del predicado q/2 ni siquiera ha llegado a ejecutarse
porque el corte ha comprometido el resultado del objetivo al resultado
de Y>15 en la primera cláusula.


\subsection*{Usos del corte}
\label{sec:org4ed8b78}

El corte se utiliza muy frecuentemente, cuanto más diestro es el
programador más lo suele usar. Los motivos por los que se usa el corte
son, por orden de importancia, los siguientes:

\begin{itemize}
\item Para optimizar la ejecución. El corte sirve para evitar que por
culpa del backtracking se exploren puntos de elección que, con toda
seguridad, no llevan a otra solución (fallan). Para los entendidos,
esto es podar el árbol de búsqueda de posibles soluciones.
\item Para facilitar la legibilidad y comprensión del algoritmo que está
siendo programado. A veces se situan cortes en puntos donde, con
toda seguridad, no van a existir puntos de elección para eliminar,
pero ayuda a entender que la ejecución sólo depende de la cláusula
en cuestión.
\item Para implementar algoritmos diferentes según la combinación de
argumentos de entrada. Algo similar al comportamiento de las
sentencias case en los lenguajes imperativos.
\item Para conseguir que un predicado solamente tenga una solución. Esto
nos puede interesar en algún momento. Una vez que el programa
encuentra una solución ejecutamos un corte. Así evitamos que Prolog
busque otras soluciones aunque sabemos que éstas existen.
\end{itemize}

\subsection*{Corte y fallo}
\label{sec:orga58f9c5}
Es muy habitual encontrar la secuencia de objetivos corte-fallo:
!,fail. El predicado fail/0 es un predicado predefinido que siempre
falla. Se utiliza para detectar prematuramente combinaciones de los
argumentos que no llevan a solución, evitando la ejecución de un
montón de código que al final va a fallar de todas formas.

\begin{minted}[]{prolog}
fib(0, 1) :- !.
fib(1, 1) :- !.
fib(N, F) :-
        N > 1,
        N1 is N-1,
        N2 is N-2,
        fib(N1, F1),
        fib(N2, F2),
        F is F1+F2.
\end{minted}
\end{document}